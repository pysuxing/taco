\section{Introduction}\label{sec:intro}
Dense linear algebra libraries are the most fundamental software in
scientific and engineering computing domains.
Basic Linear Algebra Subprograms (BLAS) (FIXME) defines a collection
of APIs which act as standard building blocks for dense matrix operations.
BLAS routines are divided into 3 levels,
level-1 for vector-vector operations,
level-2 for matrix-vetor operations,
and level-3 for matrix-matrix operations.
Processor vendors often provide BLAS implementations
highly optimized for their processors,
such as Intel MKL, AMD ACML and NVIDIA cuBLAS.
The HPC community has also contributed several high-quality
BLAS implementations, e.g. ATLAS~\cite{atlas},
GotoBLAS~\cite{gotoblas}, OpenBLAS~\cite{openblas},
and BLIS~\cite{blis}\cite{blisport}.

Among the three levels of BLAS routines, level-3 provides the most opportunities
for optimization because it has the highest computational complexity ($O(n^3)$).
Among all the level-3 operations, GEneral Matrix Multiply (GEMM) is
of the most interest because all other level-3 operations can be defined
in terms of GEMM and some level-1 and level-2 operations.
Much effort has been spent on optimizing GEMM for different architectures.
%% both in general algorithm and in architecture specific ways.
(FIXME citations)

GEMM performs a matrix-multiply-accumulate operation.
Generally, an optimized GEMM implementation consists of two components,
(1) a highly optimized kernel routine to accomplish the computation, and
(2) an overall strategy to partition the workload into small tasks
and schedule these tasks to be executed effectively on the target processor.
The kernel routine is a serial program performing matrix-multiply-accumulate
on matrices from a single task.
The overall strategy partitions the workload by tiling the loops,
and determines a task schedule by choosing a specific traversal order
in the loop nest's iteration space.
In a multi-threaded context, different tasks (loop tiles) are assigned to
different threads to parallelize the whole operation.
The main target of GEMM optimization is maximizing the floating-point
operation throughput, measured by FLoating-point Operations Per Second ($flops$).
On one hand, optimization of the serial kernel routine concentrates on
instruction scheduling to improve instruction throughput.
On the other hand, the overall strategy is responsible for optimizing
memory accesses to protect the kernel from being idling waiting for data.
In this paper, we focus on optimizing the memory access of GEMM.

Currently, two techniques, software prefetching and data packing,
are widely adopted in GEMM implementations to speed up memory accesses.
Software prefetch instructions are utilized
to load data into cache before they are used, so memory accesses will not
cause long latencies and floating-point instructions can execute at peak throughput.
To better exploit the capability of on-chip caches,
matrix blocks the kernel operates on will be packed into continous memory buffer
whose footprint is within the cache capacity before the kernel routine is called.
Because GEMM is a computational intensive operation whose
arithmetic complexity is $O(n^3)$ and memory complexity is $O(n^2)$,
the overhead caused by data packing is negligible when
the input matrices are large~\cite{gotogemm}.

The software prefetching and data packing approach meet a challenge
when working on architectures with non-LRU shared caches.
In traditional high-performance processors, e.g. Intel Xeon series,
both L1 and L2 caches are private to processor cores
and LRU replacement policy is adopted.
Without considering the last level cache (LLC),
threads running on different cores prefetch their data into
different caches, and private data of different threads
cannot cause conflict cache misses.
This is not the case if caches are shared by several processor cores.
There are chances that a cache line prefetched by one thread
get evicted out before its lifetime ends because another thread prefetches
another cache line into the same cache set.
A simple solution would be shrinking the size of packed matrix
so data used by several threads can simultaneously reside in the shared cache.
Due to the nature of set-associative caches,
this solution cannot completely eliminate inter-thread cache conflicts,
especially if the shared cache has a non-LRU replacement policy.
Evaluation results show that inter-thread cache conflicts on non-LRU shared caches
can heavily harm GEMM performance,
even with packed matrices shrinked to fit int the cache.

Following the trend in recent decades, modern architecture design
is introducing more and more cores on a single processor chip.
To reduce cost of on-chip cache memories and coherence networks,
shared cache may become common in future many-core processors.
And non-LRU replacement policies, for example, psudo-random,
may also be used to further reduce the cache design complexity.
Indeed, some high-performance processors based on ARM technology,
have already adopted this design. (FIXME list the processors)
As a consequence, a better solution to the inter-thread cache conflicts problem
on non-LRU shared caches is meaningful.

In this paper, we present a Shared Cache Partitioning (SCP) method,
to reduce inter-thread cache conflicts on architectures with non-LRU share caches.
The key idea is partition the share cache into physically disjoint sets
and assign different sets to different threads. This can be achieved by
exploiting the memory mapping mechanisim of set-associative caches.

The main contributions of this paper are:
\begin{itemize}
\item We present a quantitative analysis of the negative effect of inter-thread cache
  conflicts on GEMM performance.
\item We propose SCP, a method for solving the inter-thread cache conflicts
  problem on architectures with non-LRU shared caches.
\item We implement SCP in the OpenBLAS library and evaluate it on Phytium 2000+,
  an emerging high-performance processor based on ARM's AArch64 architecture,
  which has 64 cores, private LRU L1 caches and psudo-random shared L2 caches.
  Evaluation results show that GEMM performance improves consistently
  under various parallelism configurations, by an amount of $2.75\%$--$6.91\%$..
\end{itemize}

To the best of our knowledge, SCP is the first proposal targeted at
solving the inter-thread cache conflicts problem on
architectures with non-LRU share caches.

The rest of the paper is organized as follows.
Section~\ref{sec:background} gives some background on GEMM and
the cache conflicts problem in GEMM problem.
Section~\ref{sec:scp} illustrates the SCP method.
Section~\ref{sec:evaluation} presents evaluation results.
Section~\ref{sec:related} reviews the related work.
Finally, Section~\ref{sec:conclusion} concludes.

\section{SCP: The Shared Cache Partitioning Method}\label{sec:scp}

This section demonstrates the SCP method to handle inter-thread cache conflicts.
The basic idea is that inter-thread cache conflicts can be eliminated
if packed matrices used by different threads are fetched to different cache sets.
In Section~\ref{subsec:example}, we demonstrate SCP by an example.
Then a formal description is given in Section~\ref{subsec:formal}.

\subsection{Example}\label{subsec:example}

Suppose we are implementing a double precision GEMM (DGEMM) on
the 4-core processor, i.e. $es=8B$.
We now determine the tiling factors used in DGEMM following the procedure described
by constraints (\ref{eq:constraints.reg})--(\ref{eq:constraints.l3}).
Various configurations for ($M_r$, $N_r$) can satisfy (\ref{eq:constraints.reg}),
e.g. $(4,8)$, $(8,4)$, $(6,8)$, and $(8,6)$. 
It takes careful consideration to determine which one is the best,
and a tuning process may be needed.
After tuning we obtain the best values $M_r = 4$ and $N_r = 8$.
By (\ref{eq:constraints.l1}),
$K_c \le \lfloor c_{L1}/es/nt_{L1}/(N_r + 2 M_r) \rfloor = \lfloor 32KB/8B/1/16 \rfloor = 256$.
Here we choose $K_c=256$.
By (\ref{eq:constraints.l2}),
$M_c \le \lfloor (c_{L2}/es/nt_{L2} - 2 N_r K_c )/ K_c \rfloor =
\lfloor (2MB/8B/4 - 2*8*256)/256 \rfloor = 240$.
To leave space for other data structures, as well as the program code,
it is reasonable to shrink $M_c$ to a slightly smaller value,
e.g. $192$, $208$ and $224$ are all proper candidates for $M_c$.
Similar to $M_r$ and $N_r$, a tunning process may be needed to
find the best value for $M_c$.
Here we choose $M_c = 192$.
As there is no L3 cache, (\ref{eq:constraints.l3}) can be ignored
and $N_c$ can be given a large value.
Here we choose $N_c = 1024NT = 4096$.
With tiling factors determined, the sizes of packed matrices
can be calculated. Table.~\ref{tab:msizes} list the sizes of
various packed matrices.

\begin{table}
  \centering
  \caption{Sizes of packed matrices in DGEMM ($es = 8B$)}
  \label{tab:msizes}
  \begin{tabular}{cl|cl}
    \toprule
    matrix & size & matrix & size \\
    \midrule
    %% $A_2$, $A_3$ & $es M_c K_c = 384KB$ & $B_2$ & $es N_c K_c = 8MB$ \\
    %% $A_4$ & $es M_r K_c = 8KB$   & $B_3$, $B_4$ & $es N_r K_c = 16KB$ \\
    $A_2$ & $es M_c K_c = 384KB$ & $B_2$ & $es N_c K_c = 8MB$ \\
    $A_3$ & $es M_c K_c = 384KB$ & $B_3$ & $es N_r K_c = 16KB$ \\
    $A_4$ & $es M_r K_c = 8KB$   & $B_4$ & $es N_r K_c = 16KB$ \\
    \bottomrule
  \end{tabular}
\end{table}

\begin{figure}
  \centering
  \subfigure[Way-partitioning]{
    \label{fig:partition.conventional}
    \includegraphics[width=.4\textwidth]{figures/wpart}
  }
  \subfigure[Set-partitioning]{
    \label{fig:partition.segmented}
    \includegraphics[width=.4\textwidth]{figures/spart}
  }
  \caption{Partition styles of the shared L2 cache (n=16, ns=2048)}
  \label{fig:partition}
\end{figure}

Each $A_2$ is packed into a single continous buffer.
Figure.~\ref{fig:partition.conventional} shows how the
$A_2$ instances of 4 threads are distributed in the shared 16-way L2 cache.
We see that every cache set $s \in [0,2048)$ contains
data from all 4 $A_2$ instances,
potentially leading to inter-thread cache conflicts.
What if the $A_2$ matrices are distributed in the way shown in
Figure.~\ref{fig:partition.segmented}?
$A_2$ of different threads live in strictly disjoint cache sets,
and inter-thread cache conflicts caused by $A_2$ are eliminated completely!

Figure.~\ref{fig:partition.conventional} and Figure.~\ref{fig:partition.segmented}
essentially represent two different partitioning styles of the shared cache,
refered to as way-partitioning and set-partitioning, respectively.
To guarantee the data layout in set-partitioning,
$A_2$ cannot be stored in a single continous buffer any more.
Instead, $A_2$ is distributed to 12 continous memory segments, each of size $32KB$.
The distance from one segment to the next is equal to $wc_2=128KB$.
%% There is no memory fragment because the segment size is a multiple of
%% size of $A_4$ (8KB).
%% FIXME put this to the end of this section
While the set-partitioning of shared caches avoids inter-thread cache conflicts,
it introduces some complexity to GEMM implementation
because the GEBP kernel and the matrix packing routines
now must work with segmented memory buffers instead of continous ones.

How about $B_2$, the other packed matrix use in GEBP?
Unlike $A_2$, which is thread private,
$B_2$ is shared among all threads and
the packing of $B_2$ is done collabratively by all threads.
We have two choices, privatize $B_2$ and apply the set-partitioning method,
or fall back to conventional way-partitioning for $B_2$.
$B_2$ can be made thread private if every thread makes
its own pack of the whole $B_1$ matrix.
The first choice achieves a full per-thread data isolation
at the expense of redundant packing overhead.
The second choice avoids the extra overhead in both time and space
but risks inter-thread conflicts caused by the shared $B_2$ matrix.
The two choices are evaluated and compared in Section~\ref{sec:evaluation}.

\subsection{Formal Description}\label{subsec:formal}
Given tiling factors $M_r$, $N_r$, $K_c$, $M_c$, $N_c$,
and architectural details of the memory hierachy,
The SCP method systematically determines the
the memory layout for packed matrices $A_2$ and $B_2$ used in GEMM.
Memory layout of other packed matrices on lower layers are
determined by $A_2$ and $B_2$.
We assume that the share matrix $B_2$ is packed
in conventional way-partitioning style.
If desired, $B_2$ can be privatized and handled in the same way as $A_2$.

Before diving into details, we make a few assumptions,
which are standard practice rarely violated,
on the architectures that SCP can deal with:
\begin{itemize}
\item The architecture has 2 or 3 levels of cache. %% i.e. $NL=2$ or $NL=3$
\item All caches are inclusive set-associative caches. %% i.e. $n_l \ge 2$ for $l \in [1, NL]$
\item Caches on the same level are homogeneous. %% i.e. $L_l^i = L_l^j$ for $i \ne j$, $l \in [1, NL]$.
\item A cache's way-capacity is a multiple of the sum of way-capacity of its children,
i.e. $(\frac{nt_l}{nt_{l-1}} \cdot wc_{l-1}) \mid wc_l$.
\end{itemize}

We use a memory descriptor $\mathcal{D}_t^M$ to specify 
the memory space occupied by packed matrix $M$ used by thread $t$.
The memory descriptor is essentially a subspace of
the whole address space, i.e. $\mathcal{D}_t^M \subset \mathcal{A}$.
%% The memory layout of a packed matrix is described by a memory descriptor,
%% which is essentially a subspace of the whole address space $\mathcal{A}$.
%% Each packed matrix $M$ used by thread $t$,
%% is associated with a memory descriptor $\mathcal{D}_t^M$
%% specifying the memory space it occupies.
Generally, $\mathcal{D}_t^M$ consists of a sequence of $N_t^M$ disjoint memory segments,
$\mathcal{D}_t^M = \{ S_0, S_1, \cdots, S_{N_t^M}\}$.
%% in which $S_i \bigcap S_j = \phi$ if $i \ne j$.
A segment is represented by its start and end addresses $S_i = [s_i, e_i)$,
in which $s_i < e_i \in \mathcal{A}$.
With way-partitioning, the memory descriptor contains only one segment.
With set-partitioning used in previous section,
the memory descriptor contains a sequence of equally strided segments.

%% FIXME use l as the level iterator?
\begin{algorithm}
  \caption{SCP phase 1: allocate memory buffers}
  \label{alg:scp.phase1}
  \begin{algorithmic}[1]
    \REQUIRE $L_l = (c_l,l_l,n_l,nt_l)$ for $l \in [0, NL]$, $NT$,
    $M_r$, $N_r$, $K_c$, $M_c$, $N_c$
    \ENSURE $\mathcal{D}_p$ and $\mathcal{D}_s$
    \STATE $size_p \gets es M_c K_c$ \label{line:size.p}
    \STATE $size_s \gets es N_c K_c / NT$ \label{line:size.s}
    \STATE $align \gets l_1$ \label{line:align.init}
    \FOR {$l=1$ to $NL$} \label{line:align.for}
    \IF {$nt_l / nt_{l-1} > 1$ \AND $L_l$ is non-LRU} \label{line:align.type}
    \STATE $align \gets lcm(align, wc_l / nt_l)$ \label{line:align.update}
    \ENDIF
    \ENDFOR \label{line:align.endfor}
    \STATE $size_p \gets \lceil size_p / align \rceil \cdot align$ \label{line:align}
    %% \STATE $size \gets NT \cdot size_p + size_s$
    \STATE $addr_p \gets allocate(\mathcal{A}, size_p \cdot NT)$ \label{line:alloc.begin}
    \STATE $addr_s \gets allocate(\mathcal{A}, size_s \cdot NT)$ \label{line:alloc.end}
    \STATE $\mathcal{D}_p \gets \lbrace [addr_p, addr_p + size_p \cdot NT) \rbrace$ \label{line:d.begin}
    \STATE $\mathcal{D}_s \gets \lbrace [addr_s, addr_s + size_s \cdot NT) \rbrace$ \label{line:d.end}
  \end{algorithmic}
\end{algorithm}

SCP runs in three phases.
The first phase allocates memory buffers for all $A_2$ and $B_2$
instances used in GEMM. There are two buffers, $\mathcal{D}_p$ and $\mathcal{D}_s$,
for storing private and shared matrices respectively.
The second phase computes memory descriptors for $A_2$ instances
by partitioning $\mathcal{D}_p$ into $NT$ thread-private buffers
$\mathcal{D}_t^{A_2}$ , $t \in [0, NT)$.
Buffers of different threads are disjoint,
i.e. $\mathcal{D}_i^{A_2} \bigcap \mathcal{D}_j^{A_2} = \phi$ if $i \ne j$.
The third phase computes memory descriptors for $B_2$.
Because is $B_2$ is shared among all threads,
$\mathcal{D}_t^{B_2}$ does not represent the whole $B_2$ matrix,
but the part (of size $\frac{es N_c K_c}{NT}$) which is packed by thread $t$.
The working flow of these three phases is listed
in Algorithm~\ref{alg:scp.phase1} -- \ref{alg:scp.phase3}.

In Algorithm~\ref{alg:scp.phase1},
the per-thread size for memory buffer $\mathcal{D}_p$ and
$\mathcal{D}_s$ are first computed
(line~\ref{line:size.p}--\ref{line:size.s}).
Because the memory space in $\mathcal{D}_p$ will be set-partitioned,
extra efforts are made to ensure a proper alignment for $\mathcal{D}_p$.
Line~\ref{line:align.init}--\ref{line:align.endfor} compute a value $align$,
and $size_p$ is enlarged to a multile of $align$ (line~\ref{line:align}).
$align$ is initialized with the L1 cache line size $l_1$.
Then on each level $l$ (line~\ref{line:align.for})
with shared non-LRU caches (line~\ref{line:align.type}),
$align$ is updated by computing the least-common-multiple of
original $align$ and $wc_l/nt_l$ (line~\ref{line:align.update}).
The insight in the $wc_l/nt_l$ value is that
each way of the level-$l$ cache is equally partitioned among $nt_l$ threads
in the set-partitioning of shared caches.
At last, memory is allocated (line~\ref{line:alloc.begin}--\ref{line:alloc.end})
and buffers $\mathcal{D}_p$ and $\mathcal{D}_s$
are created (line~\ref{line:d.begin}--\ref{line:d.end}).

\begin{algorithm}
  %% a trick to temporally define customized command
  \renewcommand{\algorithmicprint}{\textbf{call}}
  \renewcommand{\algorithmicwhile}{\textbf{procedure}}
  \renewcommand{\algorithmicendwhile}{\textbf{end procedure}}
  \caption{SCP phase 2: compute descriptors for $A_2$}
  \label{alg:scp.phase2}
  \begin{algorithmic}[1]
    \REQUIRE $L_l = (c_l,l_l,n_l,nt_l)$ for $l \in [0, NL]$, $\mathcal{D}_p$, $NT$
    \ENSURE $\mathcal{D}_t^{A_2}$ for $t \in [0, NT)$
    %% \ENSURE $\mathcal{D}_l^i$ for $l \in [0, NL]$ and $i \in [0, NT/nt_l]$
    \STATE $\mathcal{D}_{NL+1}^0 \gets \mathcal{D}_p$ \label{line:memory.d}
    \STATE $nt_{NL+1} \gets NT$ \label{line:memory.nt}
    \PRINT $subspace(NL+1, 0, \mathcal{D}_{NL+1}^0)$ \label{line:subspace.root}
    \STATE                      % for a blank line
    \WHILE {$subspace\ (l, idx, \mathcal{D}_l^{idx})\ $} \label{line:subspace.begin}
    \IF {$l = 0$} \label{line:shortcut.begin}
    \STATE $\mathcal{D}_{idx}^{A_2} \gets \mathcal{D}_l^{idx}$
    \RETURN
    \ENDIF \label{line:shortcut.end}
    \STATE $nchildren \gets nt_l / nt_{l-1}$ \label{line:nchildren}
    \FOR {$i = 0$ to $nchildren-1$} \label{line:for.begin}
    \STATE $cidx \gets idx \cdot nchildren + i$ \label{line:cidx}
    \IF {$l = NL+1$ \OR $L_l$ is LRU} \label{line:if}
    \STATE $lb \gets min(\mathcal{D}_l^{idx})$
    \STATE $ub \gets max(\mathcal{D}_l^{idx})+1$
    \STATE $len \gets (ub - lb) / nchildren$
    \STATE $\mathcal{D}_{temp} \gets [lb + i \cdot len, lb + (i+1) \cdot len)$
    \ELSE \label{line:else}
    \STATE $len \gets ns_{l} / nchildren$
    \STATE $\mathcal{S}_i \gets [i \cdot len, (i+1) \cdot len)$
    \STATE $\mathcal{D}_{temp} \gets \varphi_l^{-1}(\mathcal{S}_i)$
    \ENDIF \label{line:endif}
    \STATE $\mathcal{D}_{l-1}^{cidx} \gets \mathcal{D}_{l}^{idx} \bigcap \mathcal{D}_{temp}$
    \label{line:childspace}
    \PRINT $subspace(l-1, cidx, \mathcal{D}_{l-1}^{cidx})$ \label{line:recursive}
    \ENDFOR \label{line:for.end}
    \ENDWHILE \label{line:subspace.end}
  \end{algorithmic}
\end{algorithm}

%% FIXME memory buffer? memory space? subspace?
Algorithm~\ref{alg:scp.phase2} shows the second phase of SCP.
The main part is a recursive procedure $subspace$
(line~\ref{line:subspace.begin}--\ref{line:subspace.end}).
The $subspace$ procedure traverses the memory hierachy
to compute for each node, including the processor cores and main memory,
a subspace of $\mathcal{D}_p$.
$subspace$ has three input parameters, $l$ and $idx$ to
identify the node on which it is running,
and a subspace $\mathcal{D}_l^{idx} \subset \mathcal{D}_p$ allocated to the node. 
The functionality of $subspace$ is to partition $\mathcal{D}_l^{idx}$
among all children of node $(l, idx)$.
If $subspace$ encounters a level 0 node, i.e. a processor core,
then $\mathcal{D}_l^{idx}$ is exactly the thread-private
memory buffer $\mathcal{D}_{idx}$ for thread $idx$ and $subspace$ returns early
(line~\ref{line:shortcut.begin}--\ref{line:shortcut.end}).
If this is not the case, the number of children is computed (line~\ref{line:nchildren})
and $subspace$ iterates over all the child nodes
(line~\ref{line:for.begin}--\ref{line:for.end}).
For each child node $cidx$ on layer $l-1$ (line~\ref{line:cidx}),
its memory space $\mathcal{D}_{l-1}^{cidx}$ is computed
and $subspace$ is called recursively on it (line~\ref{line:recursive}).
$\mathcal{D}_{l-1}^{cidx} \subset \mathcal{D}_l^{idx}$ is obtained
by intersecting $\mathcal{D}_l^{idx}$ with
a temporal memory space $\mathcal{D}_{temp}$ (line~\ref{line:childspace}).
The if-else branches (line~\ref{line:if}--\ref{line:endif})
show two distinct ways to compute $\mathcal{D}_{temp}$.
If the node is main memory or a LRU cache (line~\ref{line:if}),
$\mathcal{D}_l^{idx}$ is partitioned in the conventional
way-partitioning style.
Otherwise, the else branch (line~\ref{line:else}) performs
a set-partitioning on $\mathcal{D}_l^{idx}$.
Initially, the main memory gets the whole $\mathcal{D}_p$ (line~\ref{line:memory.d})
and $nt_{NL+1}$ is set to $NT$ (line~\ref{line:memory.nt})
because the main memory is shared by all threads.
Then $subspace$ procedure starts from the main memory (line~\ref{line:subspace.root})
and traverses the memory hierachy in a depth-first-search order.

The computation of $\mathcal{D}_t^{B_2}$ is quite simple.
Algorithm~\ref{alg:scp.phase3} divides $\mathcal{D}_s$
into $NT$ partitions, one for each thread.
First, the range of $\mathcal{D}_s$ is obtained
(line~\ref{line:lb}--\ref{line:ub}) and
the capacity of $\mathcal{D}_s$ is equally
divided among $NT$ threads (line~\ref{line:len}).
Then $\mathcal{D}_s$ is partitioned in a way-partitioning style
and each thread $t$ get buffer space for its part in $B_2$
(line~\ref{line:thread.for}--\ref{line:thread.forend}).

\begin{algorithm}
  \caption{SCP phase 3: compute descriptors for $B_2$}
  \label{alg:scp.phase3}
  \begin{algorithmic}[1]
    \REQUIRE $L_l = (c_l,l_l,n_l,nt_l)$ for $l \in [0, NL]$, $\mathcal{D}_s$
    \ENSURE $\mathcal{D}_t^{B_2}$ for $t \in [0, NT)$
    \STATE $lb \gets min(\mathcal{D}_s)$ \label{line:lb}
    \STATE $ub \gets max(\mathcal{D}_s)+1$ \label{line:ub}
    \STATE $len \gets (ub - lb) / nchildren$ \label{line:len}
    \FOR {$t=0$ to $NT-1$} \label{line:thread.for}
    \STATE $\mathcal{D}_t^{B_2} \gets \lbrace [lb + t \cdot len, lb + (t+1) \cdot len) \rbrace$
    \ENDFOR \label{line:thread.forend}
  \end{algorithmic}
\end{algorithm}


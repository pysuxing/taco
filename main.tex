\documentclass[format=acmsmall, review=true, screen=true, anonymous=false]{acmart}

\usepackage{booktabs} % For formal tables
\usepackage{subfigure}
\usepackage{algorithmic}

\usepackage[ruled]{algorithm2e} % For algorithms
\renewcommand{\algorithmcfname}{ALGORITHM}
\SetAlFnt{\small}
\SetAlCapFnt{\small}
\SetAlCapNameFnt{\small}
\SetAlCapHSkip{0pt}
\IncMargin{-\parindent}


% Metadata Information
\acmJournal{TACO}
\acmVolume{9}
\acmNumber{4}
\acmArticle{39}
\acmYear{2010}
\acmMonth{3}
\copyrightyear{2009}
%\acmArticleSeq{9}

% Copyright
%\setcopyright{acmcopyright}
\setcopyright{acmlicensed}
%\setcopyright{rightsretained}
%\setcopyright{usgov}
%\setcopyright{usgovmixed}
%\setcopyright{cagov}
%\setcopyright{cagovmixed}

% DOI
\acmDOI{0000001.0000001}

% Paper history
\received{February 2007}
\received[revised]{March 2009}
\received[accepted]{June 2009}


% Document starts
\begin{document}
% Title portion. Note the short title for running heads
\title{SCP: Shared Cache Partitioning for High-performance GEMM}
\titlenote{New Paper, Not an Extension of a Conference Paper.}

\author{Xing Su}
\orcid{0000-0002-7514-1495}
\email{xingsu@nudt.edu.cn}
\author{Xiangke Liao}
\email{xkliao@nudt.edu.cn}
\author{Canqun Yang}
\email{cqyang@nudt.edu.cn}
\affiliation{%
  \institution{National Laboratory for Parallel and Distributed Processing,
    National University of Defense Technology}
  \streetaddress{Yanwachi Main Street 109}
  \city{Changsha}
  \state{Hunan}
  \postcode{410073}
  \country{China}}
\author{Jingling Xue}
\affiliation{%
 \institution{UNSW Sydney}
 \city{}
 \postcode{NSW 2052}
 \country{Australia}
 }
\email{jingling@cse.unsw.edu.au}

\begin{abstract}
  GEneral Matrix Multiply (GEMM) is the most fundamental
  computational kernel routine in BLAS library.
  To achieve best performance, in-memory data must be
  prefetched into fast on-chip caches ahead of use.
  Two techniques, software prefetching and data packing,
  are utilized to effectively exploit the capability of on-chip caches.
  These techniques are sufficient for lastly-recent-used (LRU) caches,
  which is often the case in traditional high-performance
  processors used in high-end servers and supercomputers.
  But in recent years the market is meeting a diversity
  in processor design, and some high-performance processors
  are equipped with shared caches with non-LRU replacement policies.
  This rises a challenge to GEMM development in multi-threaded context.
  As several threads try to load data into the shared cache simultaneously,
  inter-thread cache conflicts would increase significantly.
  We present a Shared Cache Partition (SCP) method to
  eliminate inter-thread cache conflicts in GEMM routines,
  by partitioning the shared cache into physically disjoint
  sets and assign different sets to different threads.
  We implement SCP in the OpenBLAS library and
  evaluate the performance on Phytium 2000+,
  a 64-core AArch64 processor with private LRU L1 caches
  and shared (by every 4 cores) psudo-random L2 caches.
  The results show that conflict misses on both L1 and L2
  caches are effectively reduced and GEMM performance get improved
  by $2.75\%$--$6.91\%$.
  %% FIXME BLIS? FT1500A?
\end{abstract}


%
% The code below should be generated by the tool at
% http://dl.acm.org/ccs.cfm
% Please copy and paste the code instead of the example below.
%
\begin{CCSXML}
<ccs2012>
<concept>
<concept_id>10002950.10003714.10003715.10003719</concept_id>
<concept_desc>Mathematics of computing~Computations on matrices</concept_desc>
<concept_significance>500</concept_significance>
</concept>
<concept>
<concept_id>10010520.10010521.10010528.10010536</concept_id>
<concept_desc>Computer systems organization~Multicore architectures</concept_desc>
<concept_significance>500</concept_significance>
</concept>
</ccs2012>
\end{CCSXML}

\ccsdesc[500]{Mathematics of computing~Computations on matrices}
\ccsdesc[500]{Computer systems organization~Multicore architectures}

%
% End generated code
%

\keywords{GEMM, BLAS, Optimization, Linear Algebra, High Performance Computing}

\maketitle

% The default list of authors is too long for headers.
%% \renewcommand{\shortauthors}{G. Zhou et al.}

\section{Introduction}\label{sec:intro}
Dense linear algebra libraries are the most fundamental software in
scientific and engineering computing domains.
Basic Linear Algebra Subprograms (BLAS)~\cite{blas}  defines a collection
of APIs which act as standard building blocks for dense matrix operations.
BLAS routines are divided into 3 levels,
level-1 for vector-vector operations,
level-2 for matrix-vector operations,
and level-3 for matrix-matrix operations.
Processor vendors often provide BLAS implementations
that are highly optimized for their processors,
such as Intel MKL, AMD ACML and NVIDIA cuBLAS.
The HPC community has also contributed several high-quality
BLAS implementations, e.g., ATLAS~\cite{atlas},
GotoBLAS~\cite{gotoblas}, OpenBLAS~\cite{openblas},
and BLIS~\cite{blis,blisport}.

Among the three levels of BLAS routines, level-3 provides the most opportunities
for optimization because it has the highest computational complexity (of $O(n^3)$).
Among all the level-3 operations, GEneral Matrix Multiply (GEMM) is
of the most interest because the 
other level-3 operations can be defined
in terms of GEMM and some level-1 and level-2 operations~\cite{gemmbased1}.
In the past, much effort has been spent on optimizing GEMM for different
architectures~\cite{Liu2012,Wang2015,Volkov:2008,Cui11,blispar}.
%% both in general algorithm and in architecture specific ways.

%% GEMM performs a matrix-multiply-accumulate operation.
An optimized GEMM implementation consists of two components,
(1) a highly optimized kernel routine to accomplish its
computation, and
(2) an overall strategy to partition the workload into small tasks
and schedule these tasks to be executed effectively on the target processor.
The kernel routine is a serial program performing matrix-multiply-accumulate
on matrices from a single task.
The overall strategy partitions the workload by tiling 
the underlying loop nest
and determines a task schedule by choosing a specific traversal order
in the loop nest's iteration space.
In a multi-threaded context, different tasks (loop tiles) are assigned to
different threads to parallelize the whole operation.
The main objective of GEMM optimization is to
maximize the floating-point
operation throughput, measured by FLoating-point Operations Per Second ($flops$).
For the serial kernel routine, the optimization applies
instruction scheduling to improve instruction throughput.
For the workload partitioning, the optimization 
reorganizes memory accesses to reduce long memory latencies for
the kernel.
In this paper, we focus on optimizing the memory accesses for GEMM.

Two techniques, software prefetching and data packing,
have been widely used in GEMM implementations to speed up memory accesses.
Software prefetch instructions are utilized
to load data into cache before they are used, in 
order to ensure that memory accesses will not incur
long latencies and floating-point instructions
can execute at peak throughput.
To better exploit the capability of on-chip caches,
matrices are blocked and then packed into continuous memory buffers
that can fit into the caches under consideration
before the kernel routine is called.
Because GEMM is a computational intensive operation whose
arithmetic complexity is $O(n^3)$ and memory complexity is $O(n^2)$,
the overhead caused by data packing is negligible when
the input matrices are large~\cite{gotogemm}.


However, software prefetching and data packing cannot be effectively
applied to architectures with non-LRU shared caches.
In traditional high-performance processors, e.g., the Intel Xeon series,
both L1 and L2 caches are private to processor cores
and LRU replacement policy is adopted.
Without considering the last level cache (LLC),
the threads running on different cores prefetch their data into
different caches, and private data of different threads
cannot cause conflict cache misses.
This is no longer the case if caches are shared by several processor cores.
In this case, a cache line prefetched by one thread
may be evicted before its lifetime ends because another thread prefetches
another cache line into the same cache set.
A simple-minded solution would be to reduce 
the size of a packed matrix
so that the data used by several threads can simultaneously reside in a shared cache.
Due to the nature of set-associative caches,
this solution cannot completely eliminate inter-thread cache conflicts,
especially if the shared cache has a non-LRU replacement policy.
Our evaluation shows that inter-thread cache conflicts
in non-LRU shared caches
can heavily hurt GEMM performance,
even with packed matrices reduced to fit into the cache.

Following the trend in recent decades, modern architecture design
is introducing more and more cores on a single processor chip.
To reduce cost of on-chip cache memories and coherence networks,
shared caches may become common in future many-core processors.
In addition, non-LRU replacement policies, for example, pseudo-random,
may also be used to further reduce the cache design complexity.
Indeed, some high-performance processors based on the
ARM technology, e.g., the Phytium processor series~\cite{phytium},
%% I'm sure they have shared L2 caches,
%% but have not found any technical specification about the replacement policy
%% XGene from AppliedMicro~\cite{xgene},
%% ThunderX from Cavium~\cite{thunderx}
have already adopted this design. 
As a result, developing a better solution to reduce
the inter-thread cache conflicts 
for non-LRU shared caches is important.

In this paper, we present a Shared Cache Partitioning (SCP) method
to reduce inter-thread cache conflicts on architectures with non-LRU share caches.
The key idea is to partition a
share cache into physically disjoint sets
and assign different sets to different threads. This can be achieved by
exploiting the memory mapping mechanism of set-associative caches.

To the best of our knowledge, SCP represents
the first work addressing
the inter-thread cache conflicts problem on
architectures with non-LRU share caches.

The main contributions of this paper are as follows:
\begin{itemize}
\item We present a quantitative analysis of the negative effect of inter-thread cache
  conflicts on GEMM performance.
\item We propose SCP, a method for solving the inter-thread cache conflicts
  problem on architectures with non-LRU shared caches.
\item We have implemented SCP in the OpenBLAS library and evaluated it on Phytium 2000+,
  an emerging high-performance processor based on ARM's AArch64 architecture.
  Phytium 2000+ has 64 cores, private LRU L1 caches
  and pseudo-random shared L2 caches.
  Our evaluation shows that SCP can improve the GEMM performance consistently
  under various parallelism configurations, by $2.75\%$--$6.91\%$.
\end{itemize}


The rest of the paper is organized as follows.
Section~\ref{sec:background} gives some background on GEMM and discusses
the cache conflicts incurred.
Section~\ref{sec:scp} introduces the SCP method.
Section~\ref{sec:evaluation} presents and
analyzes our experimental results.
Section~\ref{sec:related} reviews the related work.
Finally, Section~\ref{sec:conclusion} concludes.


\section{Background}\label{sec:background}

\subsection{The Memory Hierachy Review}\label{subsec:hierachy}

Modern architectures take advantage of fast on-chip caches
to overcome the ``memory wall''.
As a compromise between direct mapped cache and full-associative cache,
set-associative cache has been the dominant choice for real-world architectures.
To understand how shared cache can cause inter-thread cache conflicts
and do harm to GEMM performance, we review some details
about the working mechanisim of set-associative caches.

A set-associative cache can be defined by a 4-tuple $(c, l, n, nt)$,
in which $c$ is the cache capacity, $l$ is the cache line size,
$n$ is ways-of-associativity,
and $nt$ means the number of cores sharing the cache.
The whole cache memory is divided into $ns=c/l/n$ sets,
denoted by integral indices $\mathcal{S} = [0, ns)$,
with each set containing $n$ cache lines.
Generally, $ns$ is a power-of-two value.
An indexing function $\varphi: \mathcal{A} \mapsto \mathcal{S}$ is responsible for
mapping addresses from the address space $\mathcal{A}$
(either virtual or physical) to cache sets $\mathcal{S}$.
Data at $addr \in \mathcal{A}$ must be fetched
into the cache set $\varphi(addr) \in \mathcal{S}$ if requested.
A classic implementation of the indexing function is shown below,
%% $\varphi(addr) = (addr \gg log_2(l)) ~\&~ (ns-1)$,
with $\gg$ and $\&$ representing logical-shift-right and bitwise-and operations, respectively.
It is an extraction of the $log_2(ns)$ bits starting from the $log_2(l)$
position in $addr$'s binary form.
\begin{equation*}
  \varphi(addr) = (addr \gg log_2(l)) ~\&~ (ns-1)
  \label{eq:phi}
\end{equation*}

It is easy to prove that any continous memory block no larger than $c/n$
cannot conflict with itself.
We define a quantity, way-capacity, denoted as $wc=c/n$,
to represent this size bound for conflict-free memory blocks.

The main memory, $NL$ levels of caches, and $NT$ processor cores
form a memory hierachy generally organized into
a $NL+2$ level regular tree structure.
Processor cores at level $0$ are leaves,
caches at levels $1$ -- $NL$ are intermedia nodes
and the main memory at level $NL+1$ is the tree root.
Any node in the memory hierachy can be identified by
its layer and its unique index on the layer $(layer, index)$.
The number of nodes on layer $l$ is $NT / nt_l$.

Throughout this paper we will use a 4-core processor with
$NL=2$ levels of caches as the platform for SCP demonstration.
Figure.~\ref{fig:hierachy} shows its memory hierachy and
detailed architectural parameters are listed in Table.~\ref{tab:cluster}.
Each core has 32 128-bit (16B) vector registers,
each capable of store 2 double precision floating-point numbers.
To make a consistency, the processor cores on level 0
is represented by its registers, which can be viewed as a special
level-0 cache $L0 = (512B, 16B, 32, 1)$ with programmable replacement policy.

\begin{figure}
  \centering
  \includegraphics[width=.45\textwidth]{figures/cluster-new}
  \caption{Memory hierachy of the 4-core processor
    (nodes labeled with $(layer,index)$)}
  \label{fig:hierachy}
\end{figure}

\begin{table}
  \centering
  \caption{Architectural parameters of the 4-core processor}
  %% (see Section~\ref{subsec:hierachy} for meanings of symbols in the header row)
  \label{tab:cluster}
  \begin{tabular}{lccccccl}
    \toprule
    type & $c$ & $n$ & $l$ & $nt$ & $ns$ & $wc$ & policy \\
    \midrule
    L0 Registers  & 512B & 32 & 16B & 1 & 1 & 16B & programmable \\
    L1 Data    & 32KB & 2  & 64B & 1 & 256 & 16KB & LRU \\
    L2 Unified & 2MB  & 16 & 64B & 4 & 2048 & 128KB & psudo-random \\
    \bottomrule
  \end{tabular}
\end{table}

\subsection{Structure of GEMM}\label{subsec:gemm}

GEMM performs a matrix-multiply-accumulate operation $C = \beta C + \alpha A B$,
where $A, B$ and $C$ are matrices of sizes
$M \times K$, $K \times N$ and $M \times N$, respectively,
and $\alpha$, $\beta$ are scalars.
While this operation is algorithmically simple,
so that a 3-deep loop nest suffices to accomplish it,
a high-performance implementation can be quite
sophisticated due to the presence of multi-level memory
hierarchies on modern processors.
Figure.~\ref{fig:gemm} shows the structure of GEMM program from
the OpenBLAS library~\cite{openblas}.
Each loop (in the original 3-deep loop nest) is tiled,
resulting in a total of six loops (referred to as layers 1 -- 6).
%% Loop tiling, together with data packing and prefetching,
%% serves to improve data locality and overlap computation
%% and memory access effectively.

\begin{figure*}[t]
  \centering
  \includegraphics[width=\textwidth]{figures/gemm}
  \caption{Structure of blocked GEMM}
  \label{fig:gemm}
\end{figure*}

In this blocked algorithm, the loops over the $N$, $K$ and $M$
matrix dimensions are tiled by sizes $N_c$, $K_c$ and $M_c$ at 
layers 1, 2 and 3, respectively.
At layers 4 and 5, the $N$ and $M$ dimensions are further tiled by sizes
$N_r$ and $M_r$, respectively.
As a result, the innermost loop at layer 6 goes over the $K$
dimension for a total of $K_c$ times,
with each iteration performing a rank-1 update on
the $M_r \times N_r$ submatrix of $C$, as shown at layer 7.
In OpenBLAS, matrix $C$ is scaled by $\beta$ before the loops,
so the multiply-accumulate operation at layer 7 need not to scale $C$ again.
Tiling factors $N_c$, $K_c$, $M_c$, $N_r$ and $M_r$ are carefully selected so that
matrices on each level fit into a certain level in the memory hierachy.
Equation~(\ref{eq:constraints.reg})--(\ref{eq:constraints.l3}) show
the constraints on these tiling factors.
$es$ means the size of matrix element, e.g. 8B for double precision floating-point numbers,
and $c_l$ means the capacity of registers or caches on level $l$.
First, by (\ref{eq:constraints.reg}), $M_r$ and $N_r$ are constrained that $M_r$ elements from $A$,
$N_r$ elements from $B$ and $M_r \times N_r$ elements from $C$ can fit into registers
(the psudo level-0 cache).
Then, by (\ref{eq:constraints.l1}), $K_c$ is constrained that $B_4$ ($N_r \times K_c$),
$A_4$ ($M_r \times K_c$), and $A_4$'s counterpart in next iteration
at layer 5 fit into the L1 cache.
Then, by (\ref{eq:constraints.l2}), $M_c$ is constrained that $A_3$ ($M_c \times K_c$),
$B_3$ ($N_r \times K_c$), and $B_3$'s counterpart in next iteration
at layer 4 fit into the L2 cache.
At last, by (\ref{eq:constraints.l3}), $N_c$ is constrained that 
$B_2$ ($K_c \times N_c$) and $A_2$ ($M_c \times K_c$) fit into the L3 cache, if it exists.
Larger matrices on higher layers 1 -- 3 live in main memory.

\begin{eqnarray}
  es (M_r + N_r + M_r N_r) & \le & c_{0} \label{eq:constraints.reg}\\
  nt_{1} \cdot es (N_r K_c + 2 M_r K_c) & \le & c_{1} \label{eq:constraints.l1}\\
  nt_{2} \cdot es (M_c K_c + 2 N_r K_c) & \le & c_{2} \label{eq:constraints.l2}\\
  es (N_c K_c + nt_{3} \cdot M_c K_c)   & \le & c_{3} \label{eq:constraints.l3}
\end{eqnarray}

\begin{comment}
Table~\ref{tab:factors} shows the procedure choosing various tiling factors.
In each row, one or more tiling factors (column 1) is determined
by a size constraint (column 3) related to a certain level
in the memory hierachy (column 2).
\begin{table}
  \centering
  \caption{Procedure of choosing tiling factors}
  \label{tab:factors}
  \begin{tabular}{cccl}
    \toprule
    factor & memory & constraint \\
    \midrule
    $M_r$, $N_r$ & registers & $M_r + N_r + M_r N_r \le Regs$ \\
    $K_c$        & L1 cache  & $N_r K_c + 2 M_r K_c \le L1$ \\
    $M_c$        & L2 cache  & $M_c K_c + 2 N_r K_c \le L2$ \\
    $N_c$        & L3 cache  & $N_c K_c + M_c K_c \le L3$ \\
    \bottomrule
  \end{tabular}
\end{table}
\end{comment}

Goto~\cite{gotogemm} factors out the innermost three loops at layers 4 -- 6 for
computing $C_2\ = C_2 + \alpha A_2 B_2$ as an architecture-dependent kernel,
known as  GEBP (GEneral multiply of a Block of $A$ and a Panel of $B$).
It is worthy noting that before GEBP is called (at layer 3),
$A_1[ii:ii+M_c-1][:]$ and $B_1$ are packed into $A_2$ and $B_2$
in a special continous layout,
so consecutive memory access is ensured within the GEBP kernel.
GotoBLAS~\cite{gotoblas}, and its successor, OpenBLAS~\cite{openblas},
implement GEMM programs based on this factorization,
with GEBP highly optimized for the target processor (often coded in assembly).

As stated in Section~\ref{sec:intro}, an GEMM implementation
is made up by a kernel routine and an overall strategy.
In Figure.~\ref{fig:gemm}, GEBP surves as the kernel routine,
and the overall strategy is defined by the three outter loops at layers 1 -- 3.
Developer can control how to partition the workload by choosing
tile factors $M_c$, $N_c$ and $K_c$,
and how to schedule GEBP tasks by interchanging loop orders at layers 1 - 3.
Either the N loop at layer 1 or M loop at layer 3, or both of them,
can be parallelized in a multi-threaded context.
Specifically, the strategy in Figure.~\ref{fig:gemm} chooses a NKM loop order and
parallelizes the M loop at layer 3.
Different threads work on different parts of $A_1$
and share the same $B_1$.
As a result, there are multiple $A_2$ instances, one per thread,
and a single $B_2$ instances, shared by all threads.
To understand how the loop on layer 3 is parallelized on the 4-core processor, 
Figure.~\ref{fig:workload} shows the data accessing patterns of a single thread $T_1$.
While all color shaded submatrices are accessed by $T_1$,
only z-curve masked submatrices (a subset of color shaded ones)
are packed by the thread.
That is, each thread packs its own $A_2$ instance and 
$B_2$ is packed by all thread collabratively.
%% We will use the OpenBLAS GEMM implementation in Figure.~\ref{fig:gemm}
%% in discussion and evaluation throughout the paper.

\begin{figure}[t]
  \centering
  \includegraphics[width=.4\textwidth]{figures/workload}
  \caption{GEMM parallelization with 4 threads and data accessing patterns of thread $T_1$}
  \label{fig:workload}
\end{figure}

\subsection{Cache Conflicts in GEMM}\label{subsec:cache-conflicts}

For set-associative caches, cache misses can be categorized
into capacity misses and conflict misses.
While constraints (\ref{eq:constraints.reg}) -- (\ref{eq:constraints.l3})
serve to avoid capacity misses,
conflict misses can not be eliminated completely.
%% We use the term ``cache conflict'' to represent conflict miss events.

Recall Figure.~\ref{fig:gemm}, the packed matrix $A_2$
will be read $N_c/N_r$ times within a single GEBP execution,
and is expected to reside in L2 cache during all $N_c/N_r$ iterations at layer 4.
The packed matrix $B_4$ is similar that it is expected to
reside in L1 cache during all $M_c/M_r$ iterations at layer 5.
In general, even with carefully selected tile factors,
matrix data can still be evicted out for a few reasons.
For instance, data structures other than the packed matrices,
or program code (in case of a unified cache) may be fetched into a
cache set the matrix data lives in.
We call this kind of cache conflicts intra-thread cache conflicts
because it occurs within the execution of a single thread.
To deal with intra-thread conflicts,
the GEBP kernel repeatedly prefetch data from $A_4$ and $B_4$ to L1 cache
in every iteration at layer 6 before it actually reads them in case of
the data being evicted out accidentally.
As GEBP accesses the matrix data much more frequently than other
data structures, this repeated prefething stratecy
can remedy the penalty caused by intra-thread conflicts.

In contrary to intra-thread cache conflicts,
there exists another kind of cache conflicts,
inter-thread cache conflicts, on architectures with shared caches.
Suppose two working threads, $T_0$ and $T_1$, are running independently
on the 4-core processor in Figure.~\ref{fig:hierachy}
to accomplish a GEMM operation.
At time $t_s$, $T_0$ prefetches some data at address $addr_0$
to cache set $\varphi_{L2}(addr_0)$ of the shared L2 cache.
The data will be actually read at time $t_e$.
Then at some time $t_m$ ($t_s < t_m < t_e$),
$T_1$ prefetches data at addr $addr_1$ to L2 cache.
If $\varphi_L(addr_1)$ happens to be equal to $\varphi_2(addr_0)$,
data at $addr_1$ and $addr_0$ will be fetched to the same cache set.
As the shared L2 cache has a non-LRU policy,
data at $addr_0$ may suffer an eviction before it is actually read,
i.e. a conflict miss occurs.

In GEMM, the cache is utilized aggressively
that almost all cache space is expected to be occupied by packed matrices.
Prefetch instructions from different threads are executed repeatedly,
interleaved in an unpredictable manner.
So inter-thread cache conflicts described above
can occur frequently on non-LRU shared caches,
thus making the technique dealing with intra-thread cache conflicts helpless.
The situation goes worse with more threads sharing the same cache.



\section{SCP: The Shared Cache Partitioning Method}\label{sec:scp}

This section introduces our SCP method for reducing
inter-thread cache conflicts.
The key insight is that inter-thread cache conflicts can be eliminated
if packed matrices used by different threads are fetched to different cache sets in a shared L2 cache.
In Section~\ref{subsec:example}, we illustrate 
SCP by an example.  In Section~\ref{subsec:formal},
we give the algorithms in our SCP method.

\subsection{An Example}\label{subsec:example}

Suppose we want to implement a double precision GEMM (DGEMM) on
the 4-core processor, where $es=8B$.
We now determine the tiling factors used based on
the constraints (\ref{eq:constraints.reg}) -- (\ref{eq:constraints.l3}).
Many solutions for ($M_r$, $N_r$) can satisfy (\ref{eq:constraints.reg}), including
$(4,8)$, $(8,4)$, $(6,8)$, and $(8,6)$. Auto
tuning returns
$M_r = 4$ and $N_r = 8$ as the best.
By (\ref{eq:constraints.l1}),
$K_c \le \lfloor c_{L1}/es/nt_{L1}/(N_r + 2 M_r) \rfloor = \lfloor 32KB/8B/1/16 \rfloor = 256$.
Here, we choose $K_c=256$.
By (\ref{eq:constraints.l2}),
$M_c \le \lfloor (c_{L2}/es/nt_{L2} - 2 N_r K_c )/ K_c \rfloor =
\lfloor (2MB/8B/4 - 2*8*256)/256 \rfloor = 240$.
To leave space for other data structures and the program code,
it is reasonable to shrink $M_c$ slightly, with,
for example, $192$, $208$ and $224$ as good
candidates. Again, a turning process yields
$M_c = 192$.
As there is no L3 cache, (\ref{eq:constraints.l3}) can be ignored
and $N_c$ can be given a large value.
Here, we choose $N_c = 1024NT = 4096$.
With these tiling factors determined, the sizes of packed matrices
can be calculated. Table.~\ref{tab:msizes} lists
the sizes of various packed matrices used.

\begin{table}
  \centering
  \caption{Sizes of packed matrices in DGEMM ($es = 8B$)}
  \label{tab:msizes}
  \begin{tabular}{cl|cl}
    \toprule
    Matrix & Size & Matrix & Size \\
    \midrule
    %% $A_2$, $A_3$ & $es M_c K_c = 384KB$ & $B_2$ & $es N_c K_c = 8MB$ \\
    %% $A_4$ & $es M_r K_c = 8KB$   & $B_3$, $B_4$ & $es N_r K_c = 16KB$ \\
    $A_2$ & $es M_c K_c = 384KB$ & $B_2$ & $es N_c K_c = 8MB$ \\
    $A_3$ & $es M_c K_c = 384KB$ & $B_3$ & $es N_r K_c = 16KB$ \\
    $A_4$ & $es M_r K_c = 8KB$   & $B_4$ & $es N_r K_c = 16KB$ \\
    \bottomrule
  \end{tabular}
\end{table}

\begin{figure}
  \centering
  \subfigure[Way-partitioning]{
    \label{fig:partition.conventional}
    \includegraphics[width=.4\textwidth]{figures/wpart}
  }
  \subfigure[Set-partitioning]{
    \label{fig:partition.segmented}
    \includegraphics[width=.4\textwidth]{figures/spart}
  }
  \caption{Partition styles of the shared L2 cache (n=16, ns=2048)}
  \label{fig:partition}
\end{figure}

Each $A_2$ is packed into a single continuous buffer.
Figure~\ref{fig:partition.conventional} shows how the
$A_2$ instances of 4 threads are distributed in the shared 16-way L2 cache.
Every cache set $s \in [0,2048)$ contains
data from all 4 $A_2$ instances,
potentially leading to inter-thread cache conflicts.
What if the $A_2$ matrices are distributed in the way shown in
Figure~\ref{fig:partition.segmented}?
$A_2$ of different threads live in strictly disjoint cache sets, enabling
the inter-thread cache conflicts caused by $A_2$ to be
eliminated completely!

Figures~\ref{fig:partition.conventional} and~\ref{fig:partition.segmented}
essentially represent two different partitioning styles for a shared cache,
referred to as \emph{way-partitioning} and \emph{set-partitioning}, respectively.
To obtain the data layout in set-partitioning,
$A_2$ can no longer be stored in a single continuous buffer.
Instead, $A_2$ will be distributed to 12 continuous memory segments, each of size $32KB$.
The distance from one segment to the next is equal to $wc_2=128KB$.
%% There is no memory fragment because the segment size is a multiple of
%% size of $A_4$ (8KB).
%% FIXME put this to the end of this section
While avoiding inter-thread cache conflicts,
set-partitioning
introduces some complexity to GEMM implementations
because the GEBP kernel and the matrix packing routines
now must work with segmented memory buffers instead of continuous ones.

What about $B_2$, the other packed matrix used in GEBP?
Unlike $A_2$, which is thread-private,
$B_2$ is shared among all threads and
the packing of $B_2$ is done collaboratively by all threads.
There are two choices (with different tradeoffs), (1) privatize $B_2$ and apply 
set-partitioning and (2) 
fall back to the conventional way-partitioning 
for $B_2$.
$B_2$ can be made thread-private if every thread makes
its own pack of the whole $B_1$ matrix.
The first choice achieves a full per-thread data isolation
at the expense of redundant packing overhead.
The second choice avoids the extra overhead in both time and space
but potentially increases the inter-thread conflicts caused by the shared $B_2$ matrix.
In this work,
these two choices are evaluated and compared in Section~\ref{sec:evaluation}.

\subsection{Algorithms}\label{subsec:formal}

Given the tiling factors $M_r$, $N_r$, $K_c$, $M_c$, $N_c$,
and architectural details of the memory hierarchy,
SCP will systematically determine the
the memory layout for the
packed matrices $A_2$ and $B_2$ used in GEMM.
The memory layout of the other packed matrices at
lower layers are determined by the memory layout of
$A_2$ and $B_2$.
We assume that the share matrix $B_2$ is packed
in the conventional way-partitioning style.
If desired, $B_2$ can be privatized and handled in the same way as $A_2$.

we make a few standard assumptions, which usually
hold for the architecture considered by SCP:
\begin{itemize}
\item The architecture has 2 or 3 levels of cache; %% i.e. $NL=2$ or $NL=3$
\item All caches are inclusive set-associative caches; %% i.e. $n_l \ge 2$ for $l \in [1, NL]$
\item Caches at the same level are homogeneous; and %% i.e. $L_l^i = L_l^j$ for $i \ne j$, $l \in [1, NL]$.
\item A cache's way-capacity is a multiple of the sum of way-capacities of its children,
i.e., $(\frac{nt_l}{nt_{l-1}} \cdot wc_{l-1}) \mid wc_l$.
\end{itemize}

We use a memory descriptor $\mathcal{D}_t^M$ to specify 
the memory space occupied by a 
packed matrix $M$ used by thread $t$.
The memory descriptor is essentially a subspace of
the whole address space, i.e., $\mathcal{D}_t^M \subset \mathcal{A}$.
%% The memory layout of a packed matrix is described by a memory descriptor,
%% which is essentially a subspace of the whole address space $\mathcal{A}$.
%% Each packed matrix $M$ used by thread $t$,
%% is associated with a memory descriptor $\mathcal{D}_t^M$
%% specifying the memory space it occupies.
Generally, $\mathcal{D}_t^M$ consists of a sequence of $N_t^M$ disjoint memory segments,
$\mathcal{D}_t^M = \{ S_0, S_1, \cdots, S_{N_t^M}\}$.
%% in which $S_i \bigcap S_j = \phi$ if $i \ne j$.
A segment is represented by its start and end addresses,
$S_i = [s_i, e_i)$,
in which $s_i < e_i \in \mathcal{A}$.
With way-partitioning, the memory descriptor contains only one segment.
With set-partitioning,
the memory descriptor contains a sequence of equally strided segments.

%% FIXME use l as the level iterator?
\begin{algorithm}
  \caption{SCP phase 1: allocate memory buffers}
  \label{alg:scp.phase1}
  \begin{algorithmic}[1]
    \REQUIRE $L_l = (c_l,l_l,n_l,nt_l)$ for $l \in [0, NL]$, $NT$,
    $M_r$, $N_r$, $K_c$, $M_c$, $N_c$
    \ENSURE $\mathcal{D}_p$ and $\mathcal{D}_s$
    \STATE $size_p \gets es M_c K_c$ \label{line:size.p}
    \STATE $size_s \gets es N_c K_c / NT$ \label{line:size.s}
    \STATE $align \gets l_1$ \label{line:align.init}
    \FOR {$l=1$ to $NL$} \label{line:align.for}
    \IF {$nt_l / nt_{l-1} > 1$ \AND $L_l$ is non-LRU} \label{line:align.type}
    \STATE $align \gets lcm(align, wc_l / nt_l)$ \label{line:align.update}
    \ENDIF
    \ENDFOR \label{line:align.endfor}
    \STATE $size_p \gets \lceil size_p / align \rceil \cdot align$ \label{line:align}
    %% \STATE $size \gets NT \cdot size_p + size_s$
    \STATE $addr_p \gets allocate(\mathcal{A}, size_p \cdot NT)$ \label{line:alloc.begin}
    \STATE $addr_s \gets allocate(\mathcal{A}, size_s \cdot NT)$ \label{line:alloc.end}
    \STATE $\mathcal{D}_p \gets \lbrace [addr_p, addr_p + size_p \cdot NT) \rbrace$ \label{line:d.begin}
    \STATE $\mathcal{D}_s \gets \lbrace [addr_s, addr_s + size_s \cdot NT) \rbrace$ \label{line:d.end}
  \end{algorithmic}
\end{algorithm}

SCP runs in three phases.
The first phase allocates memory buffers for all $A_2$ and $B_2$
instances used in GEMM. There are two buffers, $\mathcal{D}_p$ and $\mathcal{D}_s$,
for storing private and shared matrices, respectively.
The second phase computes memory descriptors for $A_2$ instances
by partitioning $\mathcal{D}_p$ into $NT$ thread-private buffers
$\mathcal{D}_t^{A_2}$ , where $t \in [0, NT)$.
Buffers of different threads are disjoint,
i.e., $\mathcal{D}_i^{A_2} \bigcap \mathcal{D}_j^{A_2} = \emptyset$ if $i \ne j$.
The third phase computes memory descriptors for $B_2$.
Because is $B_2$ is shared among all threads,
$\mathcal{D}_t^{B_2}$ does not represent the whole $B_2$ matrix,
but the part (of size $\frac{es N_c K_c}{NT}$) which is packed by thread $t$.
The work flow of these three phases is listed
in Algorithms~\ref{alg:scp.phase1} -- \ref{alg:scp.phase3}.

In Algorithm~\ref{alg:scp.phase1},
the per-thread sizes for memory buffers $\mathcal{D}_p$ and
$\mathcal{D}_s$ are computed
(lines~\ref{line:size.p} -- \ref{line:size.s}).
Because the memory space in $\mathcal{D}_p$ will be set-partitioned,
extra efforts are made to ensure a proper alignment for $\mathcal{D}_p$.
Then $align$ is found
(lines~\ref{line:align.init} -- \ref{line:align.endfor}) 
and $size_p$ is enlarged to a multiple of $align$ (line~\ref{line:align}).
Specifically,
$align$ is initialized with the L1 cache line size $l_1$.
Then for each level $l$ (line~\ref{line:align.for})
with shared non-LRU caches (line~\ref{line:align.type}),
$align$ is updated by computing the least-common-multiple of
the earlier $align$ and $wc_l/nt_l$ (line~\ref{line:align.update}).
The insight on using $wc_l/nt_l$ is that
each way of the level-$l$ cache should be
equally partitioned among the $nt_l$ threads
in set-partitioning.
Finally, memory is allocated (lines~\ref{line:alloc.begin} -- \ref{line:alloc.end})
and buffers $\mathcal{D}_p$ and $\mathcal{D}_s$
are created (line~\ref{line:d.begin} -- \ref{line:d.end}).

\begin{algorithm}
  %% a trick to temporally define customized command
  \renewcommand{\algorithmicprint}{\textbf{call}}
  \renewcommand{\algorithmicwhile}{\textbf{procedure}}
  \renewcommand{\algorithmicendwhile}{\textbf{end procedure}}
  \caption{SCP phase 2: compute descriptors for $A_2$}
  \label{alg:scp.phase2}
  \begin{algorithmic}[1]
    \REQUIRE $L_l = (c_l,l_l,n_l,nt_l)$ for $l \in [0, NL]$, $\mathcal{D}_p$, $NT$
    \ENSURE $\mathcal{D}_t^{A_2}$ for $t \in [0, NT)$
    %% \ENSURE $\mathcal{D}_l^i$ for $l \in [0, NL]$ and $i \in [0, NT/nt_l]$
    \STATE $\mathcal{D}_{NL+1}^0 \gets \mathcal{D}_p$ \label{line:memory.d}
    \STATE $nt_{NL+1} \gets NT$ \label{line:memory.nt}
    \PRINT $subspace(NL+1, 0, \mathcal{D}_{NL+1}^0)$ \label{line:subspace.root}
    \STATE                      % for a blank line
    \WHILE {$subspace\ (l, idx, \mathcal{D}_l^{idx})\ $} \label{line:subspace.begin}
    \IF {$l = 0$} \label{line:shortcut.begin}
    \STATE $\mathcal{D}_{idx}^{A_2} \gets \mathcal{D}_l^{idx}$
    \RETURN
    \ENDIF \label{line:shortcut.end}
    \STATE $nchildren \gets nt_l / nt_{l-1}$ \label{line:nchildren}
    \FOR {$i = 0$ to $nchildren-1$} \label{line:for.begin}
    \STATE $cidx \gets idx \cdot nchildren + i$ \label{line:cidx}
    \IF {$l = NL+1$ \OR $L_l$ is LRU} \label{line:if}
    \STATE $lb \gets min(\mathcal{D}_l^{idx})$
    \STATE $ub \gets max(\mathcal{D}_l^{idx})+1$
    \STATE $len \gets (ub - lb) / nchildren$
    \STATE $\mathcal{D}_{temp} \gets [lb + i \cdot len, lb + (i+1) \cdot len)$
    \ELSE \label{line:else}
    \STATE $len \gets ns_{l} / nchildren$
    \STATE $\mathcal{S}_i \gets [i \cdot len, (i+1) \cdot len)$
    \STATE $\mathcal{D}_{temp} \gets \varphi_l^{-1}(\mathcal{S}_i)$
    \ENDIF \label{line:endif}
    \STATE $\mathcal{D}_{l-1}^{cidx} \gets \mathcal{D}_{l}^{idx} \bigcap \mathcal{D}_{temp}$
    \label{line:childspace}
    \PRINT $subspace(l-1, cidx, \mathcal{D}_{l-1}^{cidx})$ \label{line:recursive}
    \ENDFOR \label{line:for.end}
    \ENDWHILE \label{line:subspace.end}
  \end{algorithmic}
\end{algorithm}

%% FIXME memory buffer? memory space? subspace?
Algorithm~\ref{alg:scp.phase2} describes
the second phase of SCP.
The main component is a recursive procedure $subspace$
(lines~\ref{line:subspace.begin} -- \ref{line:subspace.end}).
The $subspace$ procedure traverses the memory hierarchy
to compute for each node, including the processor cores and main memory,
a subspace of $\mathcal{D}_p$.
This procedure
has three input parameters, $l$ and $idx$ for
identifying the node on which it is running,
and a subspace $\mathcal{D}_l^{idx} \subset \mathcal{D}_p$ allocated to the node. 
The functionality of $subspace$ is to partition $\mathcal{D}_l^{idx}$
among all the children of node $(l, idx)$.
If $subspace$ encounters a level 0 node, i.e.,
a processor core,
then $\mathcal{D}_l^{idx}$ is exactly the thread-private
memory buffer $\mathcal{D}_{idx}$ for thread $idx$,
causing $subspace$ to return early
(lines~\ref{line:shortcut.begin} -- \ref{line:shortcut.end}).
Otherwise, the number of children is computed (line~\ref{line:nchildren})
and $subspace$ iterates over all the child nodes
(lines~\ref{line:for.begin} -- \ref{line:for.end}).
For each child node $cidx$ at layer $l-1$ (line~\ref{line:cidx}),
its memory space $\mathcal{D}_{l-1}^{cidx}$ is computed
and $subspace$ is called recursively on it (line~\ref{line:recursive}).
$\mathcal{D}_{l-1}^{cidx} \subset \mathcal{D}_l^{idx}$ is obtained
by intersecting $\mathcal{D}_l^{idx}$ with
a temporal memory space $\mathcal{D}_{temp}$ (line~\ref{line:childspace}).
The if-else branch (lines~\ref{line:if} -- \ref{line:endif})
represents two distinct ways for computing
$\mathcal{D}_{temp}$.
If the node is main memory or a LRU cache (line~\ref{line:if}),
$\mathcal{D}_l^{idx}$ is partitioned in the conventional
way-partitioning style.
Otherwise, the else branch (line~\ref{line:else}) performs
a set-partitioning on $\mathcal{D}_l^{idx}$.
Initially, the main memory gets the whole $\mathcal{D}_p$ (line~\ref{line:memory.d})
and $nt_{NL+1}$ is set to $NT$ (line~\ref{line:memory.nt})
because the main memory is shared by all threads.
Then $subspace$ procedure starts from the main memory (line~\ref{line:subspace.root})
and traverses the memory hierarchy in a depth-first-search order.

The computation of $\mathcal{D}_t^{B_2}$ is quite simple.
Algorithm~\ref{alg:scp.phase3} divides $\mathcal{D}_s$
into $NT$ partitions, one for each thread.
First, the range of $\mathcal{D}_s$ is obtained
(lines~\ref{line:lb} -- \ref{line:ub}) and
the capacity of $\mathcal{D}_s$ is equally
divided among the $NT$ threads (line~\ref{line:len}).
Then $\mathcal{D}_s$ is partitioned in a way-partitioning style
and each thread $t$ get buffer space for its part in $B_2$
(line~\ref{line:thread.for}--\ref{line:thread.forend}).

\begin{algorithm}
  \caption{SCP phase 3: compute descriptors for $B_2$}
  \label{alg:scp.phase3}
  \begin{algorithmic}[1]
    \REQUIRE $L_l = (c_l,l_l,n_l,nt_l)$ for $l \in [0, NL]$, $\mathcal{D}_s$
    \ENSURE $\mathcal{D}_t^{B_2}$ for $t \in [0, NT)$
    \STATE $lb \gets min(\mathcal{D}_s)$ \label{line:lb}
    \STATE $ub \gets max(\mathcal{D}_s)+1$ \label{line:ub}
    \STATE $len \gets (ub - lb) / nchildren$ \label{line:len}
    \FOR {$t=0$ to $NT-1$} \label{line:thread.for}
    \STATE $\mathcal{D}_t^{B_2} \gets \lbrace [lb + t \cdot len, lb + (t+1) \cdot len) \rbrace$
    \ENDFOR \label{line:thread.forend}
  \end{algorithmic}
\end{algorithm}



\section{Performance Evaluation}\label{sec:evaluation}

We implement SCP in the OpenBLAS~\cite{openblas} library version 0.3.0-dev,
and evaluate it on a Phytium 2000+ processor.
The Phytium 2000+ processor is an emerging high-performance
64-core processor based on ARM's AArch64 architecture.
The 64 cores are organized into 16 clusters with each
cluster containing 4 cores.
The structure of the cluster is almost the same with
the contrived 4-core processor in Figure.~\ref{fig:hierachy}
except that L2 caches of all 16 clusters connect to the main memory
with hardware coherence.
Table.~\ref{tab:cluster} lists architectural parameters of the cores and caches.
We restrict our evaluation to DGEMM,
as in prior work~\cite{blispar,augem,poetmicro}, for two reasons.
First, the basic idea behind SCP applies to other
variants of GEMM such as SGEMM, CGEMM and ZGEMM.
Second, the LINPACK benchmark, which is used to build the
TOP500 list of world's most powerful supercomputers~\cite{top500},
relies on the DGEMM variant.

All evaluations run with the same tiling factors
$M_r$, $N_r$, $K_c$, $M_c$ and $N_c$.
The tiling factors are computed as in Section~\ref{subsec:example},
i.e., $M_r = 4$, $N_r = 8$, $K_c = 256$, $M_c = 192$, $N_c = 1024NT$.
The parallelism is controlled by two parameters,
the number of active clusters $NC$,
and the number of active threads per cluster $NT_C$.
The total number of threads is $NT = NC \cdot NT_C$.
For the Phytium 2000+ processor, the valid ranges for $NC$ and $NT_C$
are $NC \in [1, 16]$ and $NT_C \in [1, 4]$, respectively.
Generally, DGEMM performance is presented in $flops$.
%% floating-point operations per second ($flops$).
As $NT$ varies in our evaluation,
we use another metric derived from $flops$,
the average thread efficiency,
to present performance results consistently.
The average thread efficiency is a normalized value computed as
$E_{avg} = flops / (NT \cdot \widehat{flops})$
in which $\widehat{flops}$ stands for the theoretical performance peak of a single core.
%% \begin{equation*}
%%   E = flops / (NT \cdot \widehat{flops})
%% \end{equation*}

We first show how cache sharing can hurt GEMM performance
in Section~\ref{subsec:drawback}.
Then Section~\ref{subsec:benefit} demonstrates how SCP can reduce
the penalty caused by cache sharing.
Quantitative analysis of the cache miss rates
is presented in Section~\ref{subsec:analysis}.
Finally, Section~\ref{subsec:privb} discusses
the privatization of shared matrix $B_2$.
%% Three versions of DGEMM are evaluated:
%% (1) Base, the baseline version,
%% (2) SCP,  the SCP version, 
%% and (3) SCP-P, the SCP version plus privatization of shared matrix $B_2$.

\subsection{Penalty of Cache Sharing}\label{subsec:drawback}

\begin{figure}
  \centering
  \subfigure[$NT=4$]{
    \label{fig:drawback-4}
    \includegraphics[width=.31\textwidth]{figures/sec1-4}
  }
  \subfigure[$NT=8$]{
    \label{fig:drawback-8}
    \includegraphics[width=.31\textwidth]{figures/sec1-8}
  }
  \\
  \subfigure[$NT=16$]{
    \label{fig:drawback-16}
    \includegraphics[width=.31\textwidth]{figures/sec1-16}
  }
  \subfigure[$NT=32,64$]{
    \label{fig:drawback-32}
    \includegraphics[width=.31\textwidth]{figures/sec1-32}
  }
  \caption{Average thread efficiency with cache sharing}
  \label{fig:drawback}
\end{figure}

As the L2 cache is shared by cores inside the cluster,
the impact of cache sharing can be demonstrated
by comparing DGEMM performance with the same $NT$
but different $NT_C$s.
Figure.~\ref{fig:drawback} shows the evaluation results.
Each curve represents a parallelism configuration.
Different sub-figures are configured with different $NT$s,
and different curves in the same sub-figure are
configured with different $NT_C$s.
As there is only one configuration for $NT=64$ ($NC=16$, $NT_C=4$)
and no comparison is available,
results of $NT=64$ is combined with $NT=32$ in Figure.~\ref{fig:drawback-32}.

\begin{table}
  \centering
  \caption{Average thread efficiency with cache sharing}
  \label{tab:drawback}
  \begin{tabular}{cccccc}
    \toprule
     & $NT\!\!=\!\!4$ & $NT\!\!=\!\!8$ & $NT\!\!=\!\!16$ & $NT\!\!=\!\!32$ & $NT\!\!=\!\!64$ \\
    \midrule
    $NT_C\!\!=\!\!1$ & 91.46 & 91.30 & 90.30 & -     & - \\   
    $NT_C\!\!=\!\!2$ & 90.12 & 90.25 & 89.49 & 88.26 & - \\
    $NT_C\!\!=\!\!4$ & 84.05 & 80.61 & 79.74 & 79.81 & 71.71 \\
    \bottomrule
  \end{tabular}
\end{table}

From Figure.~\ref{fig:drawback} we can see that
with all $NT$s, $NT_C=1$ achieves the best performance,
and $NT_C=2$ follows, and $NT_C=4$ comes at the lowest efficiency.
Table.~\ref{tab:drawback} summarizes the average value
of $E_{avg}$ over matrix sizes.
The table can be viewed in two directions.
Vertically, results in the same column shows how $E_{avg}$ varies with $NT_C$.
With all $NT$s, $NT_C=1$ outperforms $NT_C=4$ by a large margin.
The gap is $7.4\%$ when $NT=4$ and roughly $10\%$ when $NT>4$.
Horizontally, results in the same row reflects the scalability of
DGEMM with certain $NT_C$s.
With $NT_C=1$, the performance scales linearly with the number of threads.
With $NT_C=2$, $E_{avg}$ suffers a $1.86\%$ lose when $NT$ grows from 4 to 32.
The situation is much worse for $NT_C=4$, $E_{avg}$ drops by $14.19\%$ from
$NT=4$ to $NT=64$!

Evaluation results above show that cache sharing has a considerable
impact on DGEMM performance. 

\subsection{Effectiveness of SCP}\label{subsec:benefit}

\begin{figure}
  \centering
  \subfigure[$NT=4$]{
    \label{fig:benefit-4}
    \includegraphics[width=.31\textwidth]{figures/sec2-4}
  }
  \subfigure[$NT=8$]{
    \label{fig:benefit-8}
    \includegraphics[width=.31\textwidth]{figures/sec2-8}
  }
  \subfigure[$NT=16$]{
    \label{fig:benefit-16}
    \includegraphics[width=.31\textwidth]{figures/sec2-16}
  }
  \\
  \subfigure[$NT=32$]{
    \label{fig:benefit-32}
    \includegraphics[width=.31\textwidth]{figures/sec2-32}
  }
  \subfigure[$NT=64$]{
    \label{fig:benefit-64}
    \includegraphics[width=.31\textwidth]{figures/sec2-64}
  }
  \caption{Average thread efficiency of SCP}
  \label{fig:benefit}
\end{figure}

We evaluate SCP using the same method as in Section~\ref{subsec:drawback}.
SCP is not evaluated with $NT_C=1$ because there is no need
to partition the cache if $NT_C=1$.
Figure.~\ref{fig:benefit} shows the evaluation results.
For the purpose of comparison, results without SCP
(denoted as Base in the figure) are also presented,

\begin{table}
  \centering
  \caption{Average thread efficiency improvement of SCP}
  \label{tab:win}
  \setlength{\tabcolsep}{3.5pt}
  \begin{tabular}{cccccc}
    \toprule
     & $NT=4$ & $NT=8$ & $NT=16$ & $NT=32$ & $NT=64$ \\
    \midrule
    $NT_C=2$ & 1.77 & 1.69 & 1.82 & 1.78 & - \\
    $NT_C=4$ & 2.75 & 4.62 & 4.41 & 3.11 & 6.91 \\
    \bottomrule
  \end{tabular}
\end{table}

%% \begin{table}
%%   \centering
%%   \caption{Average thread efficiency ($\%$) of SCP}
%%   \label{tab:benefit}
%%   \setlength{\tabcolsep}{3.5pt}
%%   \begin{tabular}{cccccc}
%%     \toprule
%%      & $NT\!\!=\!\!4$ & $NT\!\!=\!\!8$ & $NT\!\!=\!\!16$ & $NT\!\!=\!\!32$ & $NT\!\!=\!\!64$ \\
%%     \midrule
%%     $NT_C\!\!=\!\!2$ & 91.89/1.77 & 91.94/1.69 & 91.31/1.82 & 90.05/1.78 & - \\
%%     $NT_C\!\!=\!\!4$ & 86.80/2.75 & 85.22/4.62 & 84.15/4.41 & 82.92/3.11 & 78.62/6.91 \\
%%     \bottomrule
%%   \end{tabular}
%% \end{table}

Under all parallelism configurations, 
SCP performs better than the Base
consistently over all matrix sizes.
Table.~\ref{tab:win} summarizes the performance improvement of SCP over Base.
%% Table.~\ref{tab:benefit} summarizes the average value
%% of $E_{avg}$ in SCP evaluation.
%% There are two values in each cell, separated by a slash.
%% $E_{avg}$ stands on the left,
%% and performance win over conventional packing
%% in the corresponding parallelism configuration
%% is shown on the right for convenience.
With $NT_C=2$, SCP performs slightly better than Base
by $1.7\%$--$1.8\%$.
The performance gap becomes larger with $NT_C=4$.
ranging from $2.75\%$ to $6.91\%$.
The largest performance gain, $6.91\%$, which is a considerable improvement,
is observed under the maximum parallelism $NT=64$.
%% Looking at Table.~\ref{tab:benefit} vertically,
%% we can see that SCP still cannot fully eliminate the impact
%% of cache sharing.
%% But it do relieve the situation to a certain degree.



\subsection{Cache Miss Rate Analysis}\label{subsec:analysis}

\begin{figure}
  \centering
  \subfigure[$NT=4$]{
    \label{fig:papi-4}
    \includegraphics[width=.31\textwidth]{figures/papi-4}
  }
  \subfigure[$NT=8$]{
    \label{fig:papi-8}
    \includegraphics[width=.31\textwidth]{figures/papi-8}
  }
  \subfigure[$NT=16$]{
    \label{fig:papi-16}
    \includegraphics[width=.31\textwidth]{figures/papi-16}
  }
  \\
  \subfigure[$NT=32$]{
    \label{fig:papi-32}
    \includegraphics[width=.31\textwidth]{figures/papi-32}
  }
  \subfigure[$NT=64$]{
    \label{fig:papi-64}
    \includegraphics[width=.31\textwidth]{figures/papi-64}
  }
  \caption{Cache miss rates with $NT_C=4$}
  \label{fig:papi}
\end{figure}

To understand why SCP outperforms Base,
and whether SCP actually eliminate inter-thread conflicts,
we use the PAPI~\cite{papi} profiling tool to
analyze the cache behavior of the GEMM program.
Four cache related hardware events are counted,
which are listed and briefly described in Table.~\ref{tab:events}.
With these hardware events,
miss rates of L1 and L2 caches can be calculated with
$\frac{L1M}{L1M+L1H}$ and $\frac{L2M}{L2M+L2H}$, respectively.
The hardware events are measured in a per-thread manner
and all presented results are the average over threads.

\begin{table}
  \centering
  \caption{Hardware events in GEMM}
  \label{tab:events}
  \begin{tabular}{lll}
    \toprule
    Name & Event & Description \\
    \midrule
    L1M & PAPI\_L1\_DCM & L1 data cache misses \\
    L1H & PAPI\_L1\_DCH & L1 data cache hits \\
    L2M & PAPI\_L2\_DCM & L2 data cache misses \\
    L2H & PAPI\_L2\_DCH & L2 data cache hits \\
    \bottomrule
  \end{tabular}
\end{table}

Figure.~\ref{fig:papi} shows the miss rates of L1 and L2 data caches.
For the sake of space, we only present results with $NT_C=4$
as SCP achieves the largest improvement with full cache sharing.
Table.~\ref{tab:papi} summarizes the average value of cache miss rates
over matrix sizes.
Under all parallelism configurations,
both L1 and L2 cache miss rates are reduced with SCP applied.
The reduced amount is roughly $13\%$ for L1 and nearly $10\%$ for L2.

For the Phytium 2000+ processor whose L1 caches are private,
only L2 caches are specially handled by SCP.
From the results we see that miss rates on both L1 and L2 caches
are reduced effectively by SCP.
While the decrease of L2 miss rate is expected,
why L1 miss rate is also affected?
The reason is that the L2 cache is inclusive.
Conflict misses on the L2 cache not only evict data out from L2,
but also invalidate the evicted data's copy in all upper level L1 caches,
thus leading to more L1 cache misses.
In general, conflict misses on an inclusive cache can affect
performance of all upper level caches.

Results in this section prove that SCP can effectively
eliminate conflict misses on shared caches.
And the benefit propagates to upper level caches, too.

\begin{table}
  \centering
  \caption{Reduction of cache miss rates with $NT_C=4$}
  \label{tab:papi}
  \setlength{\tabcolsep}{3.5pt}
  \begin{tabular}{lccccc}
    \toprule
     & $NT=4$ & $NT=8$ & $NT=16$ & $NT=32$ & $NT=64$ \\
    \midrule
    L1     & 13.87 & 13.75 & 13.54 & 13.84 & 14.21 \\
    L2     & 10.00 & 10.00 & 8.27 & 8.56 & 7.62 \\
    \bottomrule
  \end{tabular}
\end{table}
%% \begin{table}
%%   \centering
%%   \caption{Cache miss rates ($\%$) with $NT_C=4$}
%%   \label{tab:papi}
%%   \setlength{\tabcolsep}{3.5pt}
%%   \begin{tabular}{lccccc}
%%     \toprule
%%      & $NT=4$ & $NT=8$ & $NT=16$ & $NT=32$ & $NT=64$ \\
%%     \midrule
%%     L1     & 7.81 & 7.83 & 7.87 & 7.88 & 7.88 \\
%%     L1 SCP & 6.73 & 6.75 & 6.81 & 6.81 & 6.81 \\
%%     L2     & 8.78 & 8.31 & 7.13 & 7.16 & 7.10 \\
%%     L2 SCP & 7.90 & 7.49 & 6.54 & 6.55 & 6.56 \\
%%     \bottomrule
%%   \end{tabular}
%% \end{table}

\subsection{Privatization of Shared Matrix}\label{subsec:privb}
In previous evaluations, SCP only set-partitions the shared cache
for $A_2$ matrices which are private to threads.
The shared $B_2$ matrix is still packed
in conventional way-partitioning style.
As mentioned in Section~\ref{subsec:example},
set-partitioning can also be applied to $B_2$ after $B_2$ is privatized.
In this section, we will discuss and evaluate this privatization alternative.

\begin{figure}
  %% FIXME draw this figure
  \centering
  \subfigure[Conventional]{
    \label{fig:packb.conventional}
    \includegraphics[width=.50\textwidth]{figures/strategy-conventional}
  }
  \subfigure[Full privatization]{
    \label{fig:packb.full}
    \includegraphics[width=.50\textwidth]{figures/strategy-full}
  }
  \subfigure[Partial privatization]{
    \label{fig:packb.partial}
    \includegraphics[width=.50\textwidth]{figures/strategy-partial}
  }
  \caption{Packing workload ($B_2$) of thread $T_1$ with $NT_C=4$ and $NC=4$}
  %% \caption{Packing strategies of shared matrix $B_2$ with $NT_C=4$ and $NC=4$}
  \label{fig:packb}
\end{figure}

Privatization introduces more packing overhead because
redundant packing of $B_2$ is enforced.
Figure.~\ref{fig:packb} shows three packing strategies for $B_2$
in a parallelism configuration $NT_C=4$ and $NC=4$.
The whole packing workload of $B_2$ is divided into 16 tasks
and the packing workload of thread $T_1$ is masked by z-curves. 
%% and the task assignment is annoted in Figure.~\ref{fig:packb}.
In conventional packing, each thread takes a single task.
In full privatization, all tasks are replicated to all threads.
As a result, the packing overhead grows in proportion to $NT$
in full privatization strategy, which is unacceptable
if GEMM is highly parallelized, e.g., $NT=64$.
Partial privatization in Figure.~\ref{fig:packb.partial}
offers an alternative to full privatization.
In partial privatization, privatization only occurs inside
a cluster so that tasks belong to one cluster need not to be
replicated to other clusters.
Consequently, extra packing overhead of the partial privatization
strategy is bounded by $NT_C$.
For instance, on Phytium 2000+, overhead of packing $B_2$
in partial privatization is limited to 4 times
of that in conventional packing.

Figure.~\ref{fig:privb} shows the evaluation results of the partial
privatization strategy with $NT_C=4$ and $NC=4$.
Results under other parallelism configurations are similar.
For the purpose of comparison,
SCP with and without privatization are both presented
(denoted as SCP-P and SCP in Figure.~\ref{fig:privb}, respectively).
Figure.~\ref{fig:privb.papi} shows the cache miss rates.
Privatization achieves slightly lower miss rates on both L1 and L2 caches.
Figure.~\ref{fig:privb.ate} shows the average thread efficiency.
Despite its lower cache miss rates,
privatization suffers a performance lose compared to SCP.
Why performance drops while cache miss rates get reduced?
The reason lies in the extra packing overhead introduced by privatization.
Figure.~\ref{fig:privb.breakdown} shows the occupancy ratio
of overall executing time of various overheads.
Overheads in GEMM includes
(1) packing of $A_2$, (2) packing of $B_2$,
and (3) synchronization.
While SCP-P involves roughly the same overheads in
packing $A_2$ and synchronization as SCP,
it spends much more time in packing $B_2$.
Table.~\ref{tab:breakdown} lists the breakdown of
executing time. We can see that overhead of packing $B_2$
is multiplied by a factor of $NT_C=4$,
and the total overhead increases from $2.35\%$ to $4.72\%$.

\begin{figure}
  \centering
  \subfigure[Cache miss rates]{
    \label{fig:privb.papi}
    \includegraphics[width=.31\textwidth]{figures/privb-papi}
  }
  \subfigure[Average thread efficiency]{
    \label{fig:privb.ate}
    \includegraphics[width=.31\textwidth]{figures/privb-ate}
  }
  \subfigure[Executing time breakdown]{
    \label{fig:privb.breakdown}
    \includegraphics[width=.65\textwidth]{figures/privb-breakdown}
  }
  \caption{Performance of partial privatization with $NT_C=4$ and $NC=4$}
  \label{fig:privb}
\end{figure}

\begin{table}
  \centering
  \caption{Executing time breakdown with $NT_C=4$ and $NC=4$}
  \label{tab:breakdown}
  \setlength{\tabcolsep}{3.5pt}
  \begin{tabular}{lcccc}
    \toprule
     & Sync & PackA & PackB & GEBP\\
    \midrule
    SCP   & 1.01 & 0.53 & 0.80 & 97.65 \\
    SCP-P & 1.00 & 0.52 & 3.20 & 95.28\\
    \bottomrule
  \end{tabular}
\end{table}

Results in this section demonstrates that
privatization of the shared matrix $B_2$ can further
eliminate inter-thread cache conflicts.
But the improvement in cache performance is very small because
$B_2$ only occupies a small portion of the shared L2 cache,
and this marginal improvement is not sufficient to
balance the extra packing overhead introduced
by privatization.


\section{Related Work}\label{sec:related}
The idea that all level-3 BLAS operations can be built 
on top of 
a high-performance GEMM implementation was first proposed
in~\cite{gemmbased1,gemmbased2}.
The GotoBLAS library~\cite{gotoblas} and its successor OpenBLAS~\cite{openblas},
are instantiated based on this insight.
So optimizing GEMM has always been the central task in developing
dense linear algebra software.
The GEMM optimization has two aspects.
One is developing fast computation kernels and
the other is choosing a proper overall workload-partitioning  strategy
to optimize memory access.

%% Broadly speaking, there are two aspects of this task,
%% developing fast computation kernels and optimizing memory access.
%% The former corresponds the to the kernel component and
%% the latter is the target of overall strategies.

There are several approaches for obtaining optimized GEMM kernels,
%% (1) assembly programming, (2) auto-tuning, and (3) directive-based programming,
yielding different tradeoffs between performance and portability.
In GotoBLAS~\cite{gotoblas}, OpenBLAS~\cite{openblas} and BLIS~\cite{blis},
the kernel component of GEMM is written by domain experts in assembly.
ATLAS~\cite{atlas} adopts the auto-tuning method to
automatically generate kernels with different parameters
in C and find the best-performing one by running them
on the actual computing system.
POET~\cite{poet,poetcgo,poetmicro} and AUGEM~\cite{augem} use a 
directive-based programming approach.
Given a GEMM kernel in C, POET inserts annotations into
the C code to direct source-to-source compiler transformations.
AUGEM uses a template-based method to match
predefined patterns in the C code and transforms the matched
C code sequence into an optimized assembly code sequence.
Both the auto-tuning and directive-based programming approaches
rely on the compiler to transform kernels in C to machine instructions.
Kernel performance can be improved with compiler techniques,
including SIMD vectorization~\cite{Larsen:2000,Zhou:2016,Zhou:2016b,
  Eichenberger2004,GCCSLP2007},
polyhedral optimization~\cite{Bondhugula2008A,Kong:2013}, 
and loop tiling~\cite{Lam1991,Spampinato:2014,Xue00}.
Recently,
\textsc{Poca}~\cite{poca} leverages a wide range of architecture-specific
abstractions available in the LLVM compiler infrastructure
and proposes a series of domain-specific yet architecture-independent optimizations
to obtain high-performance kernels in a portable way.

The overall workload-partitioning strategy used in
GEMM is mainly concerned with choosing 
tiling factors, loop orders and parallelization
techniques.
The tiling factors $M_r$, $N_r$, $M_c$, $N_c$ and $K_c$
are essential to GEMM performance.
ATLAS~\cite{atlas} relies on auto-tuning to determine optimal
values for these factors.
Analytic techniques~\cite{analytic1,analytic2,blisanalytic} 
can also be used instead of auto-tuning.
These analytic methods generally take into consideration
the cache capacity and associativity
but assume an LRU replacement policy.
As far as we know, this paper is the first to
address the problem of cache sharing with non-LRU replacement policies,
and SCP serves as the first work to solve this problem.
In \cite{gotogemm}, a detailed discussion on
choosing different loop orders is given.
GEMM are usually parallelized only at layer 3 (the $ii$ loop),
but all loops along the $M$ and $N$ matrix dimensions
are potentially parallelizable.
BLIS~\cite{blispar} allows developers to specify a sophisticated configuration
so that more than one loops at layers 1, 3, 4 and 5
can be simultaneously parallelized.
This nested parallelization is well suited to complex architecture features
like multi-sockets and hyperthreading. 



\section{Conclusion}\label{sec:conclusion}
In this paper, we present a shared cache partitioning method,
SCP, to reduce inter-thread conflicts in GEMM programs
on architectures with shared non-LRU caches.
The basic idea is to partition a shared cache into physically
disjoint sets and assign different sets to different threads.
Evaluation results show that SCP can effectively reduce
cache misses for a shared cache (and its
upper level caches),
resulting in a considerable improvement in GEMM performance
in a multi-threaded context.
%In future work, we plan to extend SCP to other dense linear
%algebra computing routines.



%% \input{samplebody-journals}

\begin{acks}

The authors would like to thank Prof. Wangqing Chi and Prof. Ruibo Wang
for providing a supported Linux kernel used in SCP evaluation.

%% This research is partly supported by
%% the National Key Research and Development Program 
%% of China (NO.2016YFB0200400), NSFC (NO.61272483, 
%% NO.61370018, NO.61402495), and Australian
%% Research Council (DP150102109 and DP170103956).

%% The work is supported by the \grantsponsor{GS501100001809}{National
%%   Natural Science Foundation of
%%   China}{http://dx.doi.org/10.13039/501100001809} under Grant
%% No.:~\grantnum{GS501100001809}{61273304\_a}
%% and~\grantnum[http://www.nnsf.cn/youngscientists]{GS501100001809}{Young
%%   Scientists' Support Program}.

\end{acks}

% Bibliography
\bibliographystyle{ACM-Reference-Format}
\bibliography{refs}

\end{document}
